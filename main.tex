\documentclass[a4paper, 10 pt, conference]{latex/ieeeconf}  % Comment this line out if you need a4paper

\IEEEoverridecommandlockouts                              % This command is only needed if
                                                          % you want to use the \thanks command
\overrideIEEEmargins                                      % Needed to meet printer requirements.


\input{./latex/filesystem/ieee_packages.tex}  % contains the latex packages for IEEEtrans
\input{./latex/filesystem/package.tex}        % contains the latex packages
% \input{./latex/filesystem/package_edition.tex}% contains the latex packages
% \usepackage[numbers]{natbib}
\input{latex/filesystem/fileSetup.tex}        % contains package and variables init.
%!TEX root = ../main.tex

%% Acronym definition example using glossaries package
%% \usepackage{acro} cannot be used with IEEEtrans use Tobias Oetiker’s {acronym}
%% The acronym environment will have a problem with IEEEtran because of the
%% modified IEEE style description list environment. The optional argument of
%% the acronym environment cannot be used to set the width of the longest label.
%% A workaround is to use \IEEEiedlistdecl to accomplish the same thing:
%%
%% \renewcommand{\IEEEiedlistdecl}{\IEEEsetlabelwidth{S
%% ONET}}
%% \begin{acronym}
%% .
%% .
%% \end{acronym}
%% \renewcommand{\IEEEiedlistdecl}{\relax}% reset back

\newacronym{psn}{PSN}{Polarized Skylight Navigation}
\newacronym{gps}{GPS}{Global Positioning System}
\newacronym{uav}{UAV}{Unmanned Aerial Vehicle}
\newacronym{dofp}{DoFP}{division-of-focal-plane}
\newacronym{aop}{AoP}{angle of polarization}
\newacronym{dopl}{DoPl}{degree of linear polarization}
\newacronym{slam}{SLAM}{simultaneous localization and mapping}
\newacronym{imu}{IMU}{inertial measurement unit}
\newacronym{gt}{GT}{ground-truth}
\newacronym{ransac}{RANSAC}{Random sample consensu}
      % contains the acronims

%% Include all macros below

\newcommand{\lorem}{{\bf LOREM}}
\newcommand{\ipsum}{{\bf IPSUM}}



\title{\LARGE \bf
Attitude estimation from polarimetric cameras}


%% \author{Olivier Morel$^{1}$% <-this % stops a space
%% \thanks{*This work was not supported by any organization}% <-this % stops a space
%% \thanks{$^{1}$ Universit\'e de Bourgogne Franche-Comt\'e,71200 Le Creusot, France
%%        {\tt\small olivier.morel@u-bourgogne.fr}}
%% }
             % contains the Title and Autor info

\begin{document}



\maketitle
\thispagestyle{empty}
\pagestyle{empty}


%%%%%%%%%%%%%%%%%%%%%%%%%%%%%%%%%%%%%%%%%%%%%%%%%%%%%%%%%%%%%%%%%%%%%%%%%%%%%%%%
\begin{abstract}
  In the robotic field, navigation and path planning applications benefit from
  a wide range of visual systems (e.g. perspective cameras, depth cameras,
  catadioptric cameras, etc.). In outdoor conditions, these systems capture
  information in which sky regions cover a major segment of the images
  acquired. However, sky regions are discarded and are not considered as visual
  cue in vision applications.  In this paper, we propose to estimate attitude
  of \gls{uav} from sky information using a polarimetric camera. Theoretically,
  we provide a framework estimating the attitude from the skylight polarized
  patterns.
  % We showcase this formulation on simulated data sets which proved
  % the benefit of using polarimetric sensors along with other visual sensors in
  % robotic applications.
  We showcase this formulation on both simulated and real-word data
  sets which proved the benefit of using polarimetric sensors along with other
  visual sensors in robotic applications.

  % Even though sky regions often covers a major segment of acquired images in
  % robotic fields. These regions usually are not used as visual cue in vision
  % applications.  In this paper we introduce how to include these regions as a
  % main source of information for attitude estimation of \gls{uav}.  Hence we
  % propose to use a polarimetric camera to measure the skylight polarized
  % patterns and a framework connecting our measurements to attitude estimation.
  % Using the framework, we report our primary results on synthetically generated
  % datasets.  The obtained results prove the capacity of polarimetric sensors
  % for attitude estimation and promote such devices to be integrated along other
  % sensors in the robotic fields.

\end{abstract}


\glsresetall % reset the acronyms from the abstract
\include*{content/intro/partI}          % the file wihtout .tex
\include*{content/intro/partIII}
\include*{content/intro/partII}
\include*{content/method/partIV}
\include*{content/experiments/exp}
\include*{content/conclusion/conclusion}



\bibliographystyle{IEEEtranS}

%\IEEEtriggeratref{17}
\bibliography{content/bib/VIPeR-biblio}

\end{document}
