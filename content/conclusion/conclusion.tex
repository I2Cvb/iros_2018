\section{Discussion and Conclusion}
\label{sec:dis-con}

This paper presents a new method to estimate attitude of a vehicle using the polarization pattern of the sky.
Contrary to conventional cameras, polarimetric cameras can exploit part of images that represent the sky and 
we have demonstrated how to take advantage of this property to estimate attitude. We first derived all equations that 
describe the relationship between the rotation matrix of the vehicle and the polarization parameters. Then the algorithm
based on the measurement of the angle of polarization of the light beam scattered by the sky is detailed leading to two approaches: absolute attitude estimation and relative attitude estimation.
Finally, in order to cope with the undesired artifacts and outliers that can occur during the measurements, a ransac model is integrated witin our framework.
Promising results were achieved after using ransac optimization and show the potential and capacity of a polarimetric camera 
to be integrated in the robotic field.
As future work, we will focus our attention to improve these preliminary results including a minimization process of the
accumulated error of prediction using filtering. Aware that the polarimetric camera cannot be a standalone system for robust attitude estimation, 
we also plan to combine the modality with geometric informations to improve the quality of the estimation.



%%% Local Variables:
%%% mode: latex
%%% TeX-master: t
%%% End:
