\section{Discussion and Conclusion}
\label{sec:dis-con}

This paper presents a new method to estimate attitude of a vehicle using the
polarization pattern of the sky.  Contrary to conventional cameras,
polarimetric cameras can exploit part of images that represent the sky and we
have demonstrated how to take advantage of this property to estimate
attitude. We first derived all equations that describe the relationship between
the rotation matrix of the vehicle and the polarization parameters.
% Then the algorithm based on the \gls{aop} measurements of the light
% beam scattered by the sky is detailed leading to two approaches: absolute
% attitude estimation and relative attitude estimation.
Herein, we proposed a model based on \gls{aop} measurements of the light beam
scattered by the sky, subsequently two approaches of the absolute attitude and
relative attitude estimation are proposed.  The former estimates the rotation
the vehicle in comparison to the origin taking into account the sun position
while the later will not consider this assumption and estimates the position of
the vehicle in the world frame considering two consecutive frames.
Finally, in order to cope with the undesired artifacts and outliers that can
occur during the measurements, a ransac model is integrated witin our
framework.  Promising results were achieved after using ransac optimization,
illustrating the potential and capacity of a polarimetric camera to be
integrated in the robotic field.  As future work, we will focus our attention
to improve these preliminary results including a minimization process of the
accumulated error of prediction using filtering. Aware that the polarimetric
camera cannot be a standalone system for robust attitude estimation, we also
plan to combine this modality with geometric informations to improve the quality
of the estimation.


\section{Acknowledgment}
\label{sec:ack}
This work is part from a project entitled VIPeR (Polarimetric Vision Applied to Robotics Nav-
igation) funded by the French National Research Agency ANR-15-CE22-0009-VIPeR.


%%% Local Variables:
%%% mode: latex
%%% TeX-master: t
%%% End:
