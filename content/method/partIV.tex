% include the figures path relative to the master file
\graphicspath{{./content/intro/figures/}}

\section{Attitude estimation}
\label{sec:g-abs-rel}
This section explains how to estimate scattering angle ($\gamma$), absolute and
relative rotation of the \gls{uav} in world frame ($R_{wv}$).

\subsection{$\gamma$ estimation}
\label{sec:gamma}
Considering that we are only measuring the angle of polarization $\alpha$
in scattering effects, we have to estimate $\gamma$ to get the vector
$v$ defined in Eq.~\ref{eq:Rwv}. This equation is valid for all the points in
sky region. However with only two celestial points, $\gamma$ can be estimated as
expressed in the following.

\begin{equation}
\begin{cases}
R^{t}\cdot s=R_{cp_{1}}\cdot\left[\begin{array}{c}
-\sin\gamma_{1}\sin\alpha_{1}\\
\sin\gamma_{1}\cos\alpha_{1}\\
\cos\gamma_{1}
\end{array}\right]\\
R^{t}\cdot s=R_{cp_{2}}\cdot\left[\begin{array}{c}
-\sin\gamma_{2}\sin\alpha_{2}\\
\sin\gamma_{2}\cos\alpha_{2}\\
\cos\gamma_{2}
\end{array}\right]
\end{cases}\label{eq:2setseq}
\end{equation}
\noindent The Eq.~\ref{eq:2setseq} can be rewritten according to:

\begin{equation}
\begin{split}
  R_{cp_{1}}\cdot
  \begin{bmatrix}
\cos\alpha_{1} & -\sin\alpha_{1} & 0\\
\sin\alpha_{1} & \cos\alpha_{1} & 0\\
0 & 0 & 1
\end{bmatrix}
\begin{bmatrix}
0\\
\sin\gamma_{1}\\
\cos\gamma_{1}
\end{bmatrix}
\\
=R_{cp_{2}}\cdot
\begin{bmatrix}
\cos\alpha_{2} & -\sin\alpha_{2} & 0\\
\sin\alpha_{2} & \cos\alpha_{2} & 0\\
0 & 0 & 1
\end{bmatrix}
\begin{bmatrix}
0\\
\sin\gamma_{2}\\
\cos\gamma_{2}
\end{bmatrix}
\end{split}
\end{equation}

leading to:
\begin{equation}
  M_{1}\cdot
  \begin{bmatrix}
0\\
\sin\gamma_{1}\\
\cos\gamma_{1}
\end{bmatrix}
=M_{2}\cdot
\begin{bmatrix}
0\\
\sin\gamma_{2}\\
\cos\gamma_{2}
\end{bmatrix}
\label{eq:2pts}
\end{equation}

\noindent By defining the matrix $M$ such
that $M=M_{2}^{t}\cdot M_{1}$, $\gamma_1$ and $\gamma_2$ is found as:
\begin{equation}
\begin{cases}
\gamma_{1}=-\arctan\frac{M_{02}}{M_{01}}\\
\gamma_{2}=-\arctan\frac{M_{20}}{M_{10}}
\end{cases}\label{eq:gamma-sol}
\end{equation}

The \gls{aop} is $2\pi$ modulus, while the $\gamma$ found in
Eq.~\ref{eq:gamma-sol} is $\pi$ modulus, this leads to two possible solutions
for vector $v$,
$\left(\alpha_{1},\gamma_{1}\right)\,\text{and}\,\left(\alpha_{1}+\pi,-\gamma_{1}\right)$

% The determination of vectors $v$ or $w$ impose to have $\alpha$
% and $\gamma$ defined respectively $\pi$ and $2\pi$ modulus or defined
% respectively $2\pi$ and $\pi$ modulus.

% Without loss of generality,
% it can be imposed that $\gamma$ (that represents the angle of scattering
% or the angle of reflection) is determined in the range $[0,\pi]$
% and $\alpha$ is determined in the range $[-\frac{\pi}{2},\frac{3\pi}{2}]$.
% Consequently, eq(\ref{eq:2pts}) must be solved considering the four
% possible cases :
% \[
% \left(\alpha_{1},\alpha_{2}\right),\,\left(\alpha_{1},\alpha_{2}+\pi\right),\,\left(\alpha_{1}+\pi,\alpha_{2}\right),\,\left(\alpha_{1}+\pi,\alpha_{2}+\pi\right).
% \]
% To compute $v$ or $w$, this can be reduced to two solutions:

% \begin{equation}
% \left(\alpha_{1},\gamma_{1}\right)\,\text{and}\,\left(\alpha_{1}+\pi,-\gamma_{1}\right).\label{eq:candidatesgamma}
% \end{equation}




\subsection{Absolute rotation}
\label{sec:abs-rot}

\subsection{Relative rotation}
\label{sec:rel-rot}



%%% Local Variables:
%%% mode: latex
%%% TeX-master: t
%%% End:
