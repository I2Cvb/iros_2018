% include the figures path relative to the master file
\graphicspath{{./content/intro/figures/}}

\section{Attitude estimation}
\label{sec:g-abs-rel}
In this section, a method is first proposed to estimate the scattering angle~$\gamma$
without any asumption regarding the constant $\rho_{lmax}$ which can not be estimated in general.
Then two approaches are considered. The first one, called absolute rotation, makes the asumption
that the sun position is knwon or can be deduced from time and the GPS location of the vehicle. The second
called relative rotation enables to estimate the relative rotation of the vehicle from the intial position of the vehicle without any additional assumption regarding the sun position.


\subsection{$\gamma$ estimation}
\label{sec:gamma}
Considering that we are only measuring the angle of polarization $\alpha$
in scattering effects, we have to estimate $\gamma$ to get the vector
$v$ defined in Eq.~\ref{eq:Rwv}. This equation is valid for all the points in
sky region. However with only two celestial points, $\gamma$ can be estimated as
expressed in the following.

\begin{equation}
\begin{cases}
R^{t}\cdot s=R_{cp_{1}}\cdot\left[\begin{array}{c}
-\sin\gamma_{1}\sin\alpha_{1}\\
\sin\gamma_{1}\cos\alpha_{1}\\
\cos\gamma_{1}
\end{array}\right]\\
R^{t}\cdot s=R_{cp_{2}}\cdot\left[\begin{array}{c}
-\sin\gamma_{2}\sin\alpha_{2}\\
\sin\gamma_{2}\cos\alpha_{2}\\
\cos\gamma_{2}
\end{array}\right]
\end{cases}\label{eq:2setseq}
\end{equation}
\noindent Using the product of $R_{cp}$ and $R_z(\alpha)$, Eq.~\ref{eq:2setseq}
is rewritten as:
% \begin{equation}
% \begin{split}
%   R_{cp_{1}}\cdot
%   \begin{bmatrix}
% \cos\alpha_{1} & -\sin\alpha_{1} & 0\\
% \sin\alpha_{1} & \cos\alpha_{1} & 0\\
% 0 & 0 & 1
% \end{bmatrix}
% \begin{bmatrix}
% 0\\
% \sin\gamma_{1}\\
% \cos\gamma_{1}
% \end{bmatrix}
% \\
% =R_{cp_{2}}\cdot
% \begin{bmatrix}
% \cos\alpha_{2} & -\sin\alpha_{2} & 0\\
% \sin\alpha_{2} & \cos\alpha_{2} & 0\\
% 0 & 0 & 1
% \end{bmatrix}
% \begin{bmatrix}
% 0\\
% \sin\gamma_{2}\\
% \cos\gamma_{2}
% \end{bmatrix}
% \end{split}
% \end{equation}

% leading to:
\begin{equation}
  M_{1}\cdot
  \begin{bmatrix}
0\\
\sin\gamma_{1}\\
\cos\gamma_{1}
\end{bmatrix}
=M_{2}\cdot
\begin{bmatrix}
0\\
\sin\gamma_{2}\\
\cos\gamma_{2}
\end{bmatrix}
\label{eq:2pts}
\end{equation}
\noindent By defining the matrix $M$ such
that $M=M_{2}^{t}\cdot M_{1}$, $\gamma_1$ and $\gamma_2$ are found as:
\begin{equation}
\begin{cases}
\gamma_{1}=-\arctan\frac{M_{02}}{M_{01}}\\
\gamma_{2}=-\arctan\frac{M_{20}}{M_{10}}
\end{cases}\label{eq:gamma-sol}
\end{equation}

The \gls{aop} is $2\pi$ modulus, while the $\gamma$ found in
Eq.~\ref{eq:gamma-sol} is $\pi$ modulus, this leads to two possible solutions
for vector $v$,
$\left(\alpha_{1},\gamma_{1}\right)\,\text{and}\,\left(\alpha_{1}+\pi,-\gamma_{1}\right)$.

\subsection{Absolute rotation}
\label{sec:abs-rot}
In order to estimate the absolute rotation and attitude of the \gls{uav}, it is
assumed that: (i) the sun position is known, (ii) using the \gls{aop} measures
of the sky (2 points), the vector $v$ is estimated, (iii) either using the \gls{aop}
from horizontally reflected areas (i.e. water) a second vectors $w$ is
estimated or the vertical in the pixel frame is known.
In this study we assumed that the vertical in the pixel frame is known, however
having any horizontal surface, the second vector can be estimated.

The aforementioned assumptions leads to the following expression, where $z$ is
the vertical in world frame ($[0, 0, 1]$) and $t$ is the time instance.
\begin{equation}
\begin{cases}
\left[s,z,s\wedge z\right] & =R(t)\cdot\left[v(t),w(t),v(t)\wedge w(t)\right]\\
 & =R_{wv}(t)\cdot R_{vc}\cdot\left[v(t),w(t),v(t)\wedge w(t)\right]
\end{cases}
\label{eq:linear_equation}
\end{equation}

Solving Eq.~\ref{eq:linear_equation} enables to get $R_{wv}(t)$. However due to
$\gamma$ ambiguities, there exist two solutions for $v$ and therefore for
$R_{wv}$.
To constraint the solutions, at each time stamp, a closest solution compare to
the previous time stamp is selected.



\subsection{Relative rotation}
\label{sec:rel-rot}

The relative rotation is estimated between two different time stamps ($t_1$,
$t_2$), Therefore Eq.~\ref{eq:linear_equation} becomes (For simplicity in the
rest, $v(t_1)$ and $v(t_2)$ are referred as $v_1$ and $v_2$.):

\begin{equation}
\begin{cases}
\left[s,z,s\wedge z\right] & =R_{wv1}\cdot R_{vc}\cdot\left[v_{1},w_{1},v_{1}\wedge w_{1}\right]\\
\left[s,z,s\wedge z\right] & =R_{wv2}\cdot R_{vc}\cdot\left[v_{2},w_{2},v_{2}\wedge w_{2}\right]
\end{cases}.
\label{eq:rel-linear_equation}
\end{equation}
\noindent which leads to:
\begin{equation}
  \begin{split}
R_{wv2}=R_{wv1}\cdot R_{vc}\cdot\left[v_{1},w_{1},v_{1}\wedge
  w_{1}\right]\cdot \\
\left[v_{2},w_{2},v_{2}\wedge w_{2}\right]^{-1} \cdot R_{vc}^{T}.
\label{eq:relative_equation}\end{split}
\end{equation}

Using the above equation, the relative rotation $R_{v_{1}v_{2}}$ is equal to:
%$=R_{wv1}^{T}\cdot R_{wv2}$ is therefore equal to:
\begin{equation}
\begin{split}
  R_{v_{1}v_{2}} &= \\
   & R_{vc}\cdot\left[v_{1},w_{1},v_{1}\wedge
  w_{1}\right]\cdot\left[v_{2},w_{2},v_{2}\wedge w_{2}\right]^{-1}\cdot
R_{vc}^{t}
\end{split}
\label{eq:final-relative}
\end{equation}


The basic theory of attitude estimation based on polarization by scattering was explained in this
section emphasizing that in both approaches, only two points from the pattern of the sky are required
to solve the scattering angle $\gamma$. As described in the following section, an algorithm to robustify
the results is implemented by considering more points.





%%% Local Variables:
%%% mode: latex
%%% TeX-master: t
%%% End:
