% include the figures path relative to the master file
%\graphicspath{ {./content/method/figures/} }

%% \section{Background}\label{sec:review}

%% Sun position, stars and sky patterns are hold as navigational cues for the centeries to come.
%% Indeed before the discovery of magnetic compass, these natural cues were the solitary sourse of navigation used by our ancestors.

%% Polarized skylight, among those, is recognized as navigation tools of some insects and animals [reference].
%% The studies shows that some insects such as desert ants (catalyphis), butterflies and dragonflies among others, despite their small brains, using the polarized light are able to navigate throuhg their paths, efficiently and robustly [refernce]. 
%% \color{blue}{It has been shown that some insects and animals, despite their small brains are able to accomplish robust and efficient navigation in complex environments.}

%% \color{red}{It need few more lines introduction about polarized skylight and the the scattering pattern of sky itself and then explanation of different methods.}




%% %%% People using ants compass 
%% %%%% Focusing on polarization compass

%% %%%% First Lambrions et al. Second Kane et al. Third Chu et al. Fourth Sarkar et al. 

%% %%%% Maybe it can start with something like this and then look at the work of the people like Chu and Sarkar.
%% %% The ability of insects to navigate effortlessly in complex environments without stressing their nervous system has been a subject of research in robotics. 
%% %% One remarkable and well studied model is that of the Saharan desert ant (Cataglyphis fortis).
%% %% Cataglyphis fortis are excellent navigators and perform mostly egocentric navigation [4].
%% %% These ants measure the direction by using skylight cues as compass, and the distance is measured by counting the step size [4]–[7].
%% %% It is further suggested by Fent [8], that besides the e-vector1 orientation of the skylight, ants also use its degree of polarization, which remains constant for a particular position of the sun and varies as the sun moves.
%% %% This makes their navigation pattern completely independent of external visual cues.

%% Lambrions \textit{et al.}~\cite{lambrinos2000insectnavigation, lambrinos1997autonomous} 
%% Polarized compass, estimation of solar azimuth
%% based on the single-scattering Rayleigh sky model which indicates:
%% Scattered light is linearly polarized perpendicular to plane containing the observer, sun and celestial observation point.  
%% A new polarized sensor that mimics the navigation strategy of a desert ant \textit{Cataglyphis}.
%% Using photosensitive diodes to percieve the polarization patterns of the sky and drive the reference heading.

%% %% Chu \textit{et al.}~\cite{chu2009application,chu2008polarizationnavigation} did the same as Lambrions using FLC to follow the path.
%% %% Polarization Compass
%% %% Designed of the actual sensor \cite{chu2008polarizationnavigation}
%% %%% Using three polarizer direction analyzer (PDA), each PDA has two POL-sensors (cross-polairzed design) and log-ration amplifier. 
%% %%% The final sensor has three PDA located at 0, 60, 120 degrees w.r.t. the sensor positive axis.
%% %% Through a fuzzy logic controller, the information is mapped to the robot's velocity.
%% %% Combined with the wheel encoders, the polarization sensor is applied in the robot automat ic navigation system.
%% %% Using their polarized sensor they estimate the absolute azimuth angle (with accuracy of $\pm 0.2^{\circ}$) that is used along with a Fuzzy Logic Controller (FLC) to navigate a mobile robot along a predifiend trajectory.


%% Usher \textit{et al.}~\cite{usher01lightcompass} porposed to use camera instead of photodiodes to cpature the Polaized pattern of sky. 

%% Ashkanazy \textit{et al.}~\cite{ashkanazy2015bio} polarized compass, estimation of solar azimuth to achieve absolute heading. Three method was proposed to estimate the solar azimuth. 
%% Fisheye camera, three camera.
%% One method to solve the ambiguity of the solutions 
%% three cameras with linear polarizer filter oriented at -60, 0, 60 degrees, respectively. 
%% Fisheye lens was used. 
%% The effects of solar elavation, cloud, polarization angle and exposure time was investigated for all the three proposed method. 
%% There is not much to say about this paper, just that they use three camera, and linear polarized filters, and estimate the solar azimuth, therefore a polarized compass.


%% Chahl \textit{et al.}~\cite{chahl2013integration} usesd three camera and investigate the polarized compass on airplane. 


%% Sarkar \textit{et al.}~\cite{sarkar2010navigationpolarized}
%% polarization compass, proposing the sensor itself, where the filters where diesinged on the CMOS detector, removing the need for rotation of the filters. 
%% The authors argue that based on their measurements the DOPL is constant for a given elevation of the sun and polarization Fresnel ratio depends on the AOP, and these information can be used for designing a compass. 
%% There is no experimental data. 
  


%% %% Miyazaki \textit{et al.}~\cite{miyazaki09sublightpolarization}
%% %% Miyazaki \textit{et al.}~\cite{miyazaki09sunlightpolarization} first to use a fisheye camera for polarimetric purposes and analysisg the polarization patterns of the sky under different weather conditions.



%% Sturlz \textit{et al.}~\cite{sturzl2012fisheye}
%% Four camera, Fisheye lens 
%% Sun azimuth and elavation both in comparison to the previous methods. (polarized compass)
%% Proposed two methods for calculating the sun azimuth and elavation. 
%% Standard deviation of elevation error = $\pm 5 ^{\circ}$ 
%% Standard deviation of azimuth error = $\pm 2.5^{\circ}$
%% experiment on the set of frames, considering a weight calcluated from the dop


%% Ma \textit{et al.}~\cite{ma2015skylightpolarization}
%% Fisheye lens, manually rotated polarizer, four rotations, 0, 45, 90 and 135.
%% The estimates the effects of the DoP on AoP in different conditions, clear sky, cloudy and partially cloudy, and they estimate the solar meridian and using that they estimate the azimuth and elavation. 

%% Miyazaki \textit{et al.}~\cite{miyazaki09sunlightpolarization}
%% Fisheye lens, three rotations  0, 45, and 90 degrees.
%% Estimaton of the solar meridian and ultimately estimation of the sun position, using neutral point (Arago and Babinet)

%% Wang \textit{et al.}~\cite{wang2014bionic} 
%% Camera based, single measurements or multiple measurements, accuracy of $0.3256^{\circ}$ only with single measurements. 
%% three camera, polarized filter at 0, 45 and 90 degrees blue filter. 

%% %% Wang \textit{et al.}~\cite{wang2015novel} 
%% %% proposed a method based on the use of polarized light sensors and geomagnetic field.
%% %% their proposed device consists of two polarized sensors, a 3-axis compass and a computer.

%% Zhang \textit{et al.}~\cite{zhang2015sky}
%% One camera and no rotating polarizer.
%% Using a handheld light field camera with wide-angle lens and triplet linear polarizer placed over its aperture stop.
%% Deisging a polarized camera suitable for skylight measurements.

%% Zhang \textit{et al.}~\cite{zhang2016robust} using s-wave plate in combination to a linear polarizer instead of triplet linear polarizer as \cite{zhang2015sky}.


%% Tang \textit{et al.}~\cite{tang2016novel}
%% polarized compass. 
%% Fisheye lens, 3 camera 0, 45, 90 degrees polarizer.
%% Pulse coupled neural network (PCNN)
%% Using DoP and AoP both. DoP was used to judge the destruction of polarized information using PCNN algorithm. 
%% Testing different conditions, namely: clear weather without shelter, clear weather with building shelter, clear weather with building and tree shelter, and cloud without shelter.
%% Under clear weather or clear weather with building shelter environments, the average error is less that 0.4 degree. 
%% Under cloudy weather the polarization mode is destroyed to some degree, however the compass information achieved was acceptable. 
%% The image processing of PCNN was not reported clearly, neither an explantion of the PCNN method. 


%% Lu \textit{et al.}~\cite{lu2015angle}
%% Polarized compass.
%% Used solar meridian as an azimuth reference and proposed a polarized navigation system.
%% Using three features of the solar meridian, $90^{\circ}$ E vector, straight line and through the principal point, they designed a two step algorithm, thresholding and Hough transform to estimate the solar azimuth. 
%% simulation and experimental results showed accuracy better than  $0.34^{\circ}$ and $0.37^{\circ}$, respectively. 




%% Hamaoui~\cite{hamaoui2017polarized}
%% I am not sure they are proposing attityude estimation. 
 
%% Only one proposing attitude estimation rather than normal compass. 
%% Polarized skylight navigation (PSN)
%% prposed to a gradient based PSN solution based on Rayleight sky model.
%% need its owen section of explanation. 
%% Camera and rotating polarizer.




%% \begin{landscape}
%% \begin{table}
%% \caption{Design summary of \gls{psn} developed sensors.}
%% \resizebox{1.0\linewidth}{!}{
%% \scriptsize{
%%   \begin{tabular}{l c cccccc}
%% \toprule
%% %% Ref &  & \multicolumn{6}{c}{Camera} \\
%% %% \cmidrule(l){3-8} 
%% Ref &  Photodiodes  & Multiple-camera &  \multicolumn{4}{c}{Singular-camera} \\
%% \cmidrule{4-8}
%%   &  & \# camera & Rotating-Pol & SLM & DoFP & LFC \\
%% \midrule
%% Lambrions \textit{et al.}~\cite{lambrinos2000insectnavigation, lambrinos1997autonomous} & $\checkmark$ \\
%% Chu \textit{et al.}~\cite{chu2009application,chu2008polarizationnavigation} & $\checkmark$\\
%% Zhao \textit{et al,}~\cite{zhao2009novel} & $\checkmark$ \\
%% Wang \textit{et al.}~\cite{wang2015novel} & $\checkmark$ \\
%% %%Usher \textit{et al.}~\cite{usher01lightcompass} &  &  & $\checkmark$-manual\\
%% Ashkanazy \textit{et al.}~\cite{ashkanazy2015bio} &  & 3  \\
%% Chahl \textit{et al.}~\cite{chahl2013integration} & $\checkmark$ &    \\
%% Sarkar \textit{et al.}~\cite{sarkar2010navigationpolarized} &  &  &  & &  $\checkmark$ \\
%% Sturlz \textit{et al.}~\cite{sturzl2012fisheye} &  & 4 \\
%% Ma \textit{et al.}~\cite{ma2015skylightpolarization} & &  & $\checkmark$ \\
%% Miyazaki \textit{et al.}~\cite{miyazaki09sunlightpolarization}  & & & $\checkmark$ \\
%% Wang \textit{et al.}~\cite{wang2014bionic} & & 3  \\
%% Zhang \textit{et al.}~\cite{zhang2015sky} & & &  



%% \bottomrule
%% \end{tabular}}}
%% \label{tab:survey-tab}
%% \end{table}


\begin{table*}
\caption{Design summary of developed \gls{psn} sensors and systems.
The current \gls{psn} systems are designed to act as polarized compass and provide the robot heading.}
\resizebox{1.0\linewidth}{!}{
\begin{threeparttable}
\scriptsize{
  \begin{tabular}{l c ccc c}
\toprule
%% Ref &  & \multicolumn{6}{c}{Camera} \\
%% \cmidrule(l){3-8} 
Ref &  Photodiodes  & \multicolumn{3}{c}{Camera} & Measurements \\
\cmidrule{3-5}
&  & Multiple(\#) &  & Singular & \\
\midrule
Lambrions \textit{et al.}~\cite{lambrinos2000insectnavigation, lambrinos1997autonomous} & $\checkmark$ & & & & 2D\tnote{1} \\
Chu \textit{et al.}~\cite{chu2009application,chu2008polarizationnavigation} & $\checkmark$ & & & & 2D \\
Zhao \textit{et al,}~\cite{zhao2009novel} & $\checkmark$ & & & & 2D \\
Wang \textit{et al.}~\cite{wang2015novel} & $\checkmark$ & & & & longtitude \& lattitude\\
%%Usher \textit{et al.}~\cite{usher01lightcompass} &  &  &  & manual-rotation\\
Chahl \textit{et al.}~\cite{chahl2013integration} & $\checkmark$ & & & & 2D   \\
Ashkanazy \textit{et al.}~\cite{ashkanazy2015bio} &  & 3  & & & 2D\\
Sarkar \textit{et al.}~\cite{sarkar2010navigationpolarized} &  &  &  & DoFP\tnote{2} & - \\
Sturlz \textit{et al.}~\cite{sturzl2012fisheye} &  & 4  & & & 3D\tnote{3}\\
Ma \textit{et al.}~\cite{ma2015skylightpolarization} &  &  &  & manual-rotation & 2D \\
Miyazaki \textit{et al.}~\cite{miyazaki09sunlightpolarization} &  &  & & manual-rotation & 2D  \\
Wang \textit{et al.}~\cite{wang2014bionic} & & 3 & & & 2D  \\
Zhang \textit{et al.}~\cite{zhang2015sky, zhang2016robust} & & & & LFC\tnote{4}& - \\
Tang \textit{et al.}~\cite{tang2016novel} & & 3 &  & - \\
Lu \textit{et al.}~\cite{lu2015angle} &  & & & auto-rotation & 2D \\
Hamaoui~\cite{hamaoui2017polarized} & & & & manual-rotation & 3D \\

\bottomrule
\end{tabular}}
\label{tab:survey-tab}
\begin{tablenotes}
\item[1] 2D measure refers to estimation of solar azimuth only. 
\item[2] Division-of-focal-plane polarimeter (micropolarizer arrays).
\item[3] 3D measure referes to estimation of solar azimuth and elevation (solar vector).
\item[4] Light Field Camera (linear polarizer triplet).
\end{tablenotes}
\end{threeparttable}}
\end{table*}




%% \end{landscape}




%% \begin{figure*}[t]
%%   \centering{
%%   \includegraphics[width=1\linewidth]{./content/method/figures/ml-2}}
%%   \caption{Common framework}
%%   \label{fig:ML-scheme}
%% \end{figure*}
 

%% \begin{figure*}[t]
%% \begin{center}
%%    \subfigure[Vitreomacular traction.]{\includegraphics[width=0.3\textwidth, height = 0.15\textheight]{./content/method/figures/Vitreomacular}}\
%%    \subfigure[Rethinal thickening.]{\includegraphics[width = 0.3\textwidth,height = 0.15\textheight]{./content/method/figures/RE}} \
%%    \subfigure[Cyst spaces, causing central and parafoveal retina thickening.]{\includegraphics[width=0.3\textwidth,height = 0.15\textheight]{./content/method/figures/Cyst}}\\
%%    \subfigure[Cyst spaces and hard exudates, causing central retinal thickening.]{\includegraphics[width = 0.3\textwidth,height = 0.15\textheight]{./content/method/figures/Cyst+HE+RE}} \
%%    \subfigure[CSR (subretinal fluid), causing central and parafoveal thickening.]{\includegraphics[width = 0.3\textwidth,height = 0.15\textheight]{./content/method/figures/CSR}} \
%%    \subfigure[CSR, hard exudates and cyst spaces.]{\includegraphics[width = 0.3\textwidth,height = 0.15\textheight]{./content/method/figures/Cyst+CSR+HE}} \\
%%    \subfigure[Cyst spaces, causing retinal thickening.]{\includegraphics[width = 0.3\textwidth,height = 0.15\textheight]{./content/method/figures/Cyst+RE}} \
%%    \subfigure[CSR and hard exudates, causing retinal thickening.]{\includegraphics[width = 0.3\textwidth,height = 0.15\textheight]{./content/method/figures/CSR+HE+RE}} \   
%%    \subfigure[Cyst spaces causing parafoveal thickening.]{\includegraphics[width = 0.3\textwidth,height = 0.15\textheight]{./content/method/figures/Cyst+RE_parafovel}} \\
    
%% \end{center}
%%     \caption{Examples of \gls{dme} cases in \gls{seri} dataset.}
%%   \label{fig:bbdd}
%% \end{figure*}



