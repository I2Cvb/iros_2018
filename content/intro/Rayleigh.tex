\section{I dont know the tite exactly}
\label{sec:rayleigh}

The unpolarized sunlight passing through our atmosphere gets scattered by
different particles within the atmosphere.  Beside deviating the direction of
propagate wave, this transition also changes the polarization state of the
incident light. This transition can be explained using Rayleigh scattering
model.  Rayleigh scattering describes the scattering of light or any
electromagnetic waves by particles much smaller than their transmission
wavelength. Accordingly it assumes that scattering particles of the atmosphere
are small, homogeneous particles much smaller than the wavelength of the
sunlight.  Despite its simplification and assumption, this model proved to be
sufficient for describing skylight scattering and polarization
patterns~\cite{pomozi2001clearsky, horvath2002ground}.

The Rayleigh model predicts that the unpolarized sunlight becomes linearly
polarized passing through the atmosphere.
Equation \ref{eq:1} shows the stokes vector for natural light after and before
passing through the atmosphere, since the last stokes parameter is $0$ after
scattering, the light is considered to be linearly polarized.
\begin{equation}
  \label{eq:1}
  s_{unp} =
  \begin{bmatrix}
    1\\0\\0\\0
  \end{bmatrix}
  \xrightarrow[]{\text{scattering}}
  s_{p}=
  \begin{bmatrix}
   s_0 \\ s_1 \\ s_2 \\ s_3
 \end{bmatrix}
\end{equation}

Having the stokes parameters the \gls{dopl} ($\rho_{l}$) in terms of stokes
parameters and the scattering angle ($\lambda$), respectively is presented as:
\begin{equation}
  \label{eq:2}
  \rho_{l} = \frac{\sqrt{s_{1}^{2}+s_{2}^{2}}}{s_0} =
  \frac{\sin^{2}(\lambda)}{\cos^{
      2}(\lambda)+1}
\end{equation}


and the \gls{dopl}, $\rho_{l}$, varies
from 0 to 1 depending on the scattering angle, $\lambda$, ($\rho_{l} = 0$ while
$\lambda = 0, \pi$ and $\rho_{l} =1$ while $\lambda =
\pi/2$)~\cite{smith2007polarization, miyazaki09sunlightpolarization}


%%% Local Variables:
%%% mode: latex
%%% TeX-master: t
%%% End:
