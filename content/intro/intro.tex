% include the figures path relative to the master file
\graphicspath{{./content/intro/figures/}}

\section{Introduction}
\label{sec:intro}

%% Scattered sunlight, due to earth's atmosphere, produces a polarized skylight
%% pattern that is symmetric about a line defined by the zenith and the current
%% sun postion.  This pattern has been used as a aide skylight polarization has
%% been used as a source of navigation

Sun position, stars and sky patterns are hold as navigational cues for
the past centuries. Indeed, before the discovery of magnetic compass,
these natural cues were the solitary source of navigation used by our
ancestors \cite{barta2005psychophysical, horvath2011trail}.  Skylight
polarized pattern that is created due to scattered sunlight, is
recognized as a navigation tool of some insects
\cite{wehner03antnavigation, labhart202odometer}.  The studies show
that some insects such as desert ants (cataglyphis), butterflies and
dragonflies among others, are able to navigate through their paths,
efficiently and robustly by using the polarized pattern of sky,
despite their small brains \cite{labhart202odometer,
  wehner03antnavigation, hamaoui2017polarized}.

Acknowledging the nature, numerous studies have been conducted on
polarized skylight pattern \cite{lambrinos2000insectnavigation,
  chu2009application, zhao2009novel,
  wang2015novel,chahl2013integration, ashkanazy2015bio,
  sturzl2012fisheye, ma2015skylightpolarization,
  miyazaki09sunlightpolarization, wang2014bionic,
  lu2015angle,hamaoui2017polarized}.  These studies, often used the
polarized pattern to create a sort of compass and estimate the solar
azimuth angle and mainly have been shared in optic filed.  Estimating
polarized patterns, however, have been a difficult and complex task.
The primary studies report the use of several photodiodes
\cite{lambrinos2000insectnavigation, chu2009application,
  zhao2009novel, wang2015novel,chahl2013integration}, while later
either multiple cameras \cite{ashkanazy2015bio, sturzl2012fisheye,
  wang2014bionic} or manual rotating filter
\cite{ma2015skylightpolarization, miyazaki09sunlightpolarization,
  lu2015angle, hamaoui2017polarized} were used.  As a consequence of
difficult and troublesome setups, exploiting the advantages of
polarized patterns in our environment have been very limited. An
example refers to the lack of using polarized sensors in \gls{uav}.

However, recent introduction of \gls{dofp} micropolarizer cameras has offered
an alternative solution \cite{nordin1999micropolarizer, nordin1999diffractive,
  millerd2006pixelated}.  In such cameras a micropolarizer filter array,
composed of a pixelated polarized filters oriented at different angles (see
Fig.~\ref{fig:dofp-sensor}), is aligned with a detector array. Thus they can
simultaneously acquire linear polarization information (i.e $S_0$, $S_1$,
$S_2$) or full stokes parameters in one image capture.

\begin{figure}
  \centering
  \includegraphics[width=0.4\textwidth]{./content/intro/figures/dofp-sensor-0-45-135-90.png}
  \label{fig:dofp-sensor}
  \caption{Structure of \gls{dofp} sensors}
\end{figure}

This study presents the primary results of using the polarization by scattering
and Rayleigh model \textcolor{red}{(Rayleigh sky scattering} for attitude estimation.
Polarization by scattering and Rayleigh model are explained in
Sect.~\ref{sec:rayleigh} while our theoretical model is presented in
Sect.~\ref{sec:attitude}.























% Some stuff that emac's colegues use
%%% Local Variables:
%%% mode: late
%%% TeX-master: "../../main.tex"
%%% End: \section{introduction}
