% include the figures path relative to the master file
\graphicspath{{./content/intro/figures/}}

\section{Introduction}
\label{sec:intro}

Adaptation of large-filed cameras and lenses such as omnidirectional and fisheye
lenses are very popular in robotic application, simply due to the provided
large field of view ($360$-degree) and extended amount of visual information.
Due to these unique properties, these cameras are applicable in a wide range of
robotics application, to name a few, visual
odometry~\cite{scaramuzza2008appearance}, navigation~\cite{winters2000omni},
\gls{slam}~\cite{kim2003slam}, and tracking~\cite{kobilarov2006people}.
Although great amount of visual information can be extracted from an omnidirectional
image, our attention in this article is on the segment that often is ignored or
not used as navigational clue, sky.
In outdoor application, sky region often covers
a large segment of omnidirectional or fisheye images and contains
information, which has not been exploited to their full extend yet.


Sun position, stars and sky patterns are hold as navigational cues for
the past centuries. Indeed, before the discovery of magnetic compass,
these natural cues were the solitary source of navigation used by our
ancestors \cite{barta2005psychophysical, horvath2011trail}.  Skylight
polarized pattern that is created due to scattered sunlight, is
recognized as a navigation tool of some insects
\cite{wehner03antnavigation, labhart202odometer}.  The studies show
that some insects such as desert ants (cataglyphis), butterflies and
dragonflies among others, are able to navigate through their paths,
efficiently and robustly by using the polarized pattern of sky,
despite their small brains \cite{labhart202odometer,
  wehner03antnavigation, hamaoui2017polarized}.


Acknowledging the nature, numerous studies have been conducted on polarized
skylight pattern \cite{lambrinos2000insectnavigation, chu2009application,
  zhao2009novel, wang2015novel,chahl2013integration, ashkanazy2015bio,
  sturzl2012fisheye, ma2015skylightpolarization,
  miyazaki09sunlightpolarization, wang2014bionic,
  lu2015angle,hamaoui2017polarized}.  These studies, often used the polarized
pattern to create a sort of compass and estimate the solar azimuth angle and
mainly have been shared in optic filed.  Estimating polarized patterns,
however, have been a difficult and complex task.  The primary studies report
the use of several photodiodes \cite{lambrinos2000insectnavigation,
  chu2009application, zhao2009novel, wang2015novel,chahl2013integration}, while
later either multiple cameras \cite{ashkanazy2015bio, sturzl2012fisheye,
  wang2014bionic} or manual rotating filter \cite{ma2015skylightpolarization,
  miyazaki09sunlightpolarization, lu2015angle, hamaoui2017polarized} were used.
As a consequence of difficult and troublesome setups, exploiting the advantages
of polarized patterns in our environment have been very limited. An example
refers to the lack of using polarized sensors in \gls{uav}.  However, recent
introduction of \gls{dofp} micropolarizer cameras has offered an alternative
solution \cite{nordin1999micropolarizer, nordin1999diffractive,
  millerd2006pixelated}.  In such cameras a micropolarizer filter array,
composed of a pixelated polarized filters oriented at different angles, is
aligned with a detector array. Thus they can simultaneously acquire linear
polarization information (i.e $S_0$, $S_1$, $S_2$) or full stokes parameters in
one image capture.
Such a camera with a fisheye lens is used in this research to exploit the
polarized information of sky region for attitude estimation.
The used camera and their adaptation for robotics application
is latter explained in Sect.~\ref{sec:p3}.

In the remainder of this paper, Sect.~\ref{sec:p2} provides a brief
introduction to polarization by scattering, Rayleigh model and explains how
these information can be used for attitude estimation. Sect.~\ref{sec:p4}
represents our model for attitude estimation. Experiments and implementation
details are explained in Sect.~\ref{sec:p5}, and finally discussion and
conclusion is presented in Sect.~\ref{sec:p6}.
























% Some stuff that emac's colegues use
%%% Local Variables:
%%% mode: late
%%% TeX-master: "../../main.tex"
%%% End: \section{introduction}
