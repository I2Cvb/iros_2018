% include the figures path relative to the master file
\graphicspath{{./content/intro/figures/}}

\section{Introduction}
\label{sec:intro}

Large-field cameras and lenses (e.g. omnidirectional and fisheye cameras) are
popular in robotic applications due to their ability to provide large field of
view (up to \ang{360}), extending the amount of visual information. It is the
main reason for which they have been adopted for a broad range of tasks such as
visual odometry~\cite{scaramuzza2008appearance},
navigation~\cite{winters2000omni}, \gls{slam}~\cite{kim2003slam}, and
tracking~\cite{kobilarov2006people}. With those systems, sky regions in the
images acquired represent a large segment of information which are usually
discarded. Here, we show that polarimetric information can be extracted from
those regions and used in robotic applications.

% Adaptation of large-filed cameras and lenses such as omnidirectional and fisheye
% lenses are very popular in robotic application, simply due to the provided
% large field of view ($360$-degree) and extended amount of visual information.
% Due to these unique properties, these cameras are applicable in a wide range of
% robotics application, to name a few, visual
% odometry~\cite{scaramuzza2008appearance}, navigation~\cite{winters2000omni},
% \gls{slam}~\cite{kim2003slam}, and tracking~\cite{kobilarov2006people}.
% Although great amount of visual information can be extracted from an omnidirectional
% image, our attention in this article is on the segment that often is ignored or
% not used as navigational clue, sky.
% In outdoor application, sky region often covers
% a large segment of omnidirectional or fisheye images and contains
% information, which has not been exploited to their full extend yet.

Sun position, stars and sky patterns are hold as navigational cues for the past
centuries. Indeed, before the discovery of magnetic compass, these natural cues
have been the solitary source of navigation used by our
ancestors~\cite{barta2005psychophysical,horvath2011trail}. Similarly, some
insects used the skylight polarized pattern created by the scattered sunlight
to navigate in their
environment~\cite{wehner03antnavigation,labhart202odometer}. For instance,
desert ants (cataglyphis), butterflies and dragonflies among others, are able
to navigate through their paths, efficiently and robustly by using the
polarized pattern of sky, despite their small
brains~\cite{labhart202odometer,wehner03antnavigation,hamaoui2017polarized}.

Acknowledging the nature, numerous studies have been conducted on polarized
skylight pattern~\cite{lambrinos2000insectnavigation, chu2009application,
  zhao2009novel, wang2015novel,chahl2013integration, ashkanazy2015bio,
  sturzl2012fisheye, ma2015skylightpolarization,
  miyazaki09sunlightpolarization, wang2014bionic,
  lu2015angle,hamaoui2017polarized}.
These studies are generally reported in the optic field. They focus on
estimating the solar azimuth angle by creating a sort of compass.
Estimating polarized patterns
have been, however, a difficult and complex task.  The primary studies report
the use of several photodiodes~\cite{lambrinos2000insectnavigation,
  chu2009application, zhao2009novel, wang2015novel,chahl2013integration}, or of multiple cameras~\cite{ashkanazy2015bio, sturzl2012fisheye,
  wang2014bionic} or manually rotating filters~\cite{ma2015skylightpolarization,
  miyazaki09sunlightpolarization, lu2015angle, hamaoui2017polarized}.
As a consequence of those troublesome setups, robotic applications are not
benefiting from the advantages of polarized patterns, as attested  by the lack
of polarized sensors used in \gls{uav}.  However, the recent
introduction of \gls{dofp} micropolarizer cameras has offered an alternative
solution~\cite{nordin1999micropolarizer, nordin1999diffractive,
  millerd2006pixelated}.  In such cameras a micropolarizer filter array,
composed of a pixelated polarized filters oriented at different angles, is
aligned with a detector array. Thus, linear polarization information are
simultaneously acquired taking a single image. Here, we use a \gls{dofp}
coupled with a fisheye lens to exploit the polarized information of sky region
to estimate vehicle attitude.

In this paper, Sect.\,\ref{sec:rosify} presents the specificity of the camera
used and the adaptation required for our robotic application. The remainder of
the paper is organized as follows: Sect.\,\ref{sec:pcues} introduces the
concepts of polarization by scattering, Rayleigh model and the relation with
attitude estimation while our formulation to estimate attitude is presented in
Sect.\,\ref{sec:g-abs-rel}. Experiments and implementation details are given
in Sect.\,\ref{sec:exp-res}, and finally discussions and conclusions are
drawn in Sect.\,\ref{sec:dis-con}.
























% Some stuff that emac's colegues use
%%% Local Variables:
%%% mode: late
%%% TeX-master: "../../main.tex"
%%% End: \section{introduction}
