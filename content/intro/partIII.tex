% include the figures path relative to the master file
\graphicspath{{./content/intro/figures/}}
\section{Setting the polarimetric camera for robotics}
\label{sec:rosify}

In this work, visual information is captured using the \emph{IMPREX Bobcat GEV}
polarimetric camera which is a \gls{dofp} polarimetric camera. In a single shot,
the camera captures four different linearly polarized measures by using
a micropolarizer with pixelated polarized filter array as illustrated in
Fig.\,\ref{fig:dofp-sensor}. Hence, each acquired image is subdivided into four
linearly polarized images $I_0$, $I_{45}$, $I_{135}$, and
$I_{90}$. Subsequently, the polarized state of the incident light is computed
from these images by means of the Stokes'
parameters~\cite{goldstein2017polarized}, referred as $s_0$, $s_1$, and $s_2$
in Eq.\,\ref{eq:stokes}. In addition, the polarized parameters \gls{aop} and
\gls{dopl}, respectively referred $\alpha$ and $\rho_l$ in
Eq.\,\ref{eq:stokes}) are computed.

% As mentioned previously, in this work we use a \gls{dofp} polarimetric camera,
% to be exact \textit{IMPREX Bobcat GEV polarimetric camera}.
% This camera has the advantage to capture four different linearly polarized
% measure in one image, due to their micropolarizer and pixelated polarized
% filter array (see Fig.\,\ref{fig:dofp-sensor}).
% Therefore each capture image leads to four linearly polarized sub-image $I_0$,
% $I_{45}$, $I_{135}$, $I_{90}$.
% Using these four measurements, the polarized state of the incident light is
% calculated in terms of stokes parameters~\cite{goldstein2017polarized}, which
% are used thereafter to calculate polarized parameters such as \gls{aop} refered
% as $\alpha$ and \gls{dopl} refered as $\rho_l$ (see Eq.\,\ref{eq:stokes}).

\begin{figure}
  \centering
  \includegraphics[width=0.3\textwidth]{./content/intro/figures/dofp-sensor-0-45-135-90.jpg}
  \caption{Structure of \gls{dofp} sensors: in a single shot, four polarized
    images are acquired, each of them with a different polarized angles.}
    \label{fig:dofp-sensor}
\end{figure}


\begin{figure*}
  \centering
  \includegraphics[width=1.0\textwidth]{./content/intro/figures/image_pola.jpeg}
  \caption{Polarimetric images: \emph{left:} a raw images in which the
    different polarimetric images are interlaced; \emph{center:} the four
    extracted linearly polarized images ($I_0, I_{45}, I_{135}, I_{90}$);
    \emph{right:} the \gls{aop} and \gls{dopl} images. For visualization purpose,
    \gls{aop} is represented in HSV colorspace.}
  \label{fig:raw-sp}
\end{figure*}


% \begin{figure*}
%   \centering
%   \includegraphics[width=0.7\textwidth]{./content/intro/figures/raw-sp.jpg}
%   \caption{A raw image captured with a fisheye lens and the four extracted
%     linearly polarized images ($I_0, I_{45}, I_{135}, I_{90}$)}
%   \label{fig:raw-sp}
% \end{figure*}

% \begin{figure}
%   \centering
%   \includegraphics[width=0.47\textwidth]{./content/intro/figures/aop-dop.jpg}
%   \caption{The \gls{aop} and \gls{dopl} images. For visualization purpose,
%     \gls{aop} is represented in HSV colorspace.
%     \label{fig:stokes-aop-dop}}
% \end{figure}


\begin{gather}
  \begin{aligned}
    % \small
    s_0 & = \frac{I_0 + I_{45} + I_{135} + I_{90}}{4}\\
    s_1 & = I_0 - I_{90} \\
    s_2 & = I_{45} - I_{135} \\
    \alpha &= 0.5 \arctan(\frac{s_2}{s_1}) \\
    \rho_l &= \frac{\sqrt{s_2^{2} + s_1^{2}}}{s_0}
    \label{eq:stokes}
  \end{aligned}
\end{gather}

The camera captures images with a resolution of $\SI{640}{\px} \times
\SI{460}{\px}$ and each polarized image is reconstructed from the interlaced
pixels. In addition, we used a \ang{180} fisheye lens for the experiment
reported in this paper to benefit from a large field-of-view. An example of an
acquired raw image, the extracted polarized images and the computed polarized
information is shown in Fig.\,\ref{fig:raw-sp}

% From the raw images ($640\times460$) captured by the camera, the sub-images can
% be extracted directly using super-pixel method that leads to four images, half
% of the size of the raw image, or can be interpolated to the full
% size~\cite{ratliff2009interpolationmicrogrid,gao2011bilinearpolarimeters}.  The
% super-pixel method was used for the results presented in this paper. Being
% interested in large-field of view, a fisheye lens of 180-degree was used on the
% camera.

% An example of captured raw image, the linearly polarized images and polarized
% information is shown in Fig.~\ref{fig:raw-sp}~$\&$~\ref{fig:aop-dop}.

The \emph{IMPREX Bobcat GEV} camera is controlled using the \emph{eBus} SDK
provided by Pleora Technologies Inc~\cite{eBus}. To enable the interaction with
other robotic devices, we implemented a ROS package publicly
available\footnote{\url{https://github.com/I2Cvb/pleora_polarcam}. This package
  is derived from earlier work~\cite{ira}.} enabling the usage of
\texttt{roslaunch} and \texttt{rosrun}. In addition, our package allows to
store and stream the raw data as well as computing the Stokes' and polarized
parameters.

% The \textit{IMPREX Bobcat GEV} camera operates using eBus SDK-pleora driver and
% libraries~\cite{eBus}. To be able to use the camera integrated with other
% sensors, in the robotic field, we have created a ROS
% package, pleora-polarcam~\cite{pleora_polarcam}.
% Initiating from Iralab photonfocus driver~\cite{ira}, pleora-polarcam package
% is adapted for Imperex polarimetric cameras.
% Using this package the user can easily \text{roslaunch} or \text{rosrun} the
% camera and beside, buffering and saving the raw data, process the stokes and
% polarized parameters.



%%%Local Variables:
%%% mode: latex
%%% TeX-master: t
%%% End:
