%!TEX root = ../main.tex

%% Acronym definition example using glossaries package
%% \usepackage{acro} cannot be used with IEEEtrans use Tobias Oetiker’s {acronym}
%% The acronym environment will have a problem with IEEEtran because of the
%% modified IEEE style description list environment. The optional argument of
%% the acronym environment cannot be used to set the width of the longest label.
%% A workaround is to use \IEEEiedlistdecl to accomplish the same thing:
%%
%% \renewcommand{\IEEEiedlistdecl}{\IEEEsetlabelwidth{S
%% ONET}}
%% \begin{acronym}
%% .
%% .
%% \end{acronym}
%% \renewcommand{\IEEEiedlistdecl}{\relax}% reset back

\newacronym{psn}{PSN}{Polarized Skylight Navigation}
\newacronym{gps}{GPS}{Global Positioning System}
\newacronym{uav}{UAV}{Unmanned Aerial Vehicle}
\newacronym{dofp}{DoFP}{division-of-focal-plane}
\newacronym{aop}{AoP}{angle of polarization}
\newacronym{dopl}{DoPl}{degree of linear polarization}
\newacronym{slam}{SLAM}{simultaneous localization and mapping}
\newacronym{imu}{IMU}{inertial measurement unit}
\newacronym{gt}{GT}{ground-truth}
\newacronym{ransac}{RANSAC}{Random sample consensu}
